\section{Introduction\label{sec:intro}}

The CMS detector currently operates at a peak instantaneous luminosity of 1$\times$10$^{34}$ cm$^{-2}$s$^{-1}$ with 25 ns bunch spacing. The instantaneous luminosity is expected to be increased by at least a factor of 2 after the detector's scheduled second long shutdown (LS2) for maintenance and upgrades in 2018. To optimize jet reconstruction and identification performance under the high pileup conditions expected, the hybrid photodiodes of the barrel (HB) and endcap (HE) regions of the hadronic calorimeter (HCAL) will be replaced during the Phase I Upgrade in early 2017 with silicon photomultipliers (SiPMs), allowing greater depth segmentation and thus increased granularity of signal detection and reconstruction.

The HCAL scintillator tiles consist of SCSN-81 plastic scintillator in a polystyrene (PS) base with Kuraray Y11 wavelength-shifting fibers. The HCAL has exhibited greater losses in light yield due to radiation damage than was previously predicted by laboratory measurements. This has led to concerns that at least part of the scintillator material will need to be replaced before the end of its originally planned lifetime, and that the present material may not be radiation-tolerant enough to survive the length of time between detector upgrades. Different kinds of plastic scintillator -- with various base, scintillator, and wavelength-shifting compounds -- are presently being investigated as candidates for the active material replacement in HE and HB during or even before LS2.

In this paper, we present an analysis of the radiation-tolerance of SCSN-81 and a few upgrade candidate materials, conducted in the high-radiation environment of the CASTOR radiation facility (CRF) located in the CMS forward region.
